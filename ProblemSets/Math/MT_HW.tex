\documentclass[14pt]{extreport}
\usepackage[]{fontenc}
\usepackage[margin=1in]{geometry}
\usepackage{amssymb, amsmath, amstext}
\usepackage{hyperref}
\usepackage{amsmath,blkarray}
\usepackage{fancyhdr}
\usepackage[]{algorithm2e}
\usepackage{setspace}
\usepackage{graphicx}
\usepackage{mathtools}
\usepackage{listings}
\usepackage{float}
\usepackage{mathtools} 
\usepackage{lscape}
\usepackage{breqn}

\parindent 0pt
\parskip 10pt


\begin{document}
\begin{center}
\large{\textbf{Measure Theory Homework}}
~\\
Terry Wu
\end{center}
\begin{spacing}{1.5}
1.3. $g_1$ is not an algebra. Take $A=(-\infty,-3) \cup (2,+\infty)$ which is in $g_1$ but $A^c=[-3,2]$ which is not in $g_1$. $g_2$ is an algebra and $g_3$ is $\sigma$-algebra.

1.7. By definition, $\emptyset$ has to be in the algebra generated by a set $X$. And by the requirement of closed under complementary, $X$ then has to be too. So the smallest algebra must be $\{\emptyset, X\}$. The largest algebra should be the one contains all possible subsets, which is the power set.

1.10. First of all, $\emptyset \in S_{\alpha}$ for all $\alpha$ because $S_{\alpha}$ is a $\sigma$-algebra. Suppose $A\in \sigma(g)$ then $A\in S_{\alpha}$ for all $\alpha$, then $A^c\in S_{\alpha}$ for all $\alpha$ by definition. Thus $A^c \in \sigma(g)$. Similarly we could prove it is closed under countable unions. 

1.22. Take $C=A^c\cap B$ then as $A\subset B$, $|C|\geq 0$. And we have $\mu(C \cup A)=\mu(B)$ which is $\mu(C)+\mu(A)=\mu(B)$, so $\mu(B)-\mu(A)\geq 0$. 

Define $F_n=\cup_{i=1}^{n}A_i$. It is easy to see that $F_n\subset F_{n+1}$. Therefore $F_n$ is increasing. Also we know that $F_n \uparrow \cup_{i=1}^{+\infty}A_i$. Hence we have
\begin{align*}
\mu(\cup_{i=1}^{+\infty}A_i)=&\lim_{n\rightarrow +\infty}\mu(F_n)\\
=&\lim_{n\rightarrow +\infty}\mu(A_1\cup A_2\cup......\cup A_n)\\
\leq & \lim_{n\rightarrow +\infty} \sum_{i=1}^{n}\mu(A_i)=  \sum_{i=1}^{+\infty}\mu(A_i).
\end{align*}

1.23. Let $A=\emptyset$, then $A\cap B=\emptyset$, so $\lambda(A)=\mu(A\cap B)=0.$ Take $A_1, A_2, ......$ to be mutually exclusive sets, then $\lambda(\cup_i A_i)=\mu((\cup_i A_i)\cap B)$ which is $\mu(\cup_i (A_i\cap B))=\sum_i \mu(A_i\cap B)=\sum_i \lambda(A_i)$. So $\lambda$ is a measure. 

1.26. Note we have $A_i \downarrow \cap_i A_i=X$. We could write $A_{n_0}=X+\sum_{n=n_0}^{+\infty}(A_{n}-A_{n+1})$. So we have $\mu(A_{n_0})=\mu(X)+\sum_{n=n_0}^{+\infty}\mu(A_n-A_{n+1})$ which completes the proof.

2.10. Because from the countable subadditivity of the outer measure we know that $\geq$ holds. As $\leq$ holds, the equal sign should also hold.

2.14. We know $o(\mathcal{O}) \subset \mathcal{M}$ from Caratheodory and construction of lebesgue measure as an infinite collection of the form $(a,b]$ and $(-\infty, a]$ we know that $\sigma(\mathcal{A})\subset \mathcal{M}$. If $o\in \sigma(\mathcal{O})$ then $o\in \sigma(\mathcal{A})$ therefore $o\in \mathcal{M}$.

3.1. Let $a_i$ be a sequence of a countable set $A$ then $A_i=(a_i-2^{-i-1}\epsilon,a_i+2^{-i-1}\epsilon)$. So $A\subset \cup A_i$ and $\mu( \cup A_i)\leq 2^{-i}\epsilon=\epsilon$ which completes the proof.

3.7. Because they are complementary to each other so the condition could be replaced. 

3.10. Let $F(f(x),g(x))=f(x)+g(x)$ is countable. Then $F$ is countable. Given condition 4 we know that $f+g$ is measurable. Similarly by taking $F(f(x),g(x))=f(x)g(x)$ we could prove that $fg$ is measurable. As $f$ and $g$ are measurable, then $\{x\in X:\max(f,g)<a\}$=$\{x\in X:g<a\}\cap\{x\in X:g<a\}$. $M$ is closed under intersection. Similarly we have $\{x\in X:\min(f,g)>a\}$=$\{x\in X:g>a\}\cup\{x\in X:g>a\}$. And finally as $\{x\in X:|f|>a\}$=$\{x\in X:f>a\}\cup\{x\in X:f<-a\}$ we could prove that $|f|$ is measurable.

4.13. We know that $||f||=f^{+}+f^{-}$. As $|f|<M$ we have $f^+\leq M$ and/or $-M<f^-$. Therefore $\int_E f^+d\mu<\infty$ and $\int_E f^-d\mu<\infty$ which completes the proof.

4.14. Suppose there exists a subset $A\subset E$ that $f$ is not finite so $f=\infty$ on $A$. Therefore $\int_E f d\mu>\int_A f d\mu + \infty$ which contradicts.

4.15. Define $s=g-f$ so we have $s\geq 0$ and $\int_E s d\mu \geq 0$. Therefore we have $\int_E (g-f) d\mu \geq 0$ which completes the proof.

4.16. From the question we know that $\int_E f^+ d\mu<\infty$ and  $\int_E f^- d\mu<\infty$. As $A\subset E$, we have $\int_A f^+ d\mu<\infty$ and  $\int_A f^- d\mu<\infty$. Therefore $f\in L^1(\mu,A)$.

4.21. Without loss, we could write $C=A \cap B^c$. So $\int_A f d\mu=\int_B f d\mu+\int_C f d\mu=\int_B f d\mu$ since $\mu(C)=0.$

\end{spacing}
\end{document}
